\documentclass[11pt]{article}
\usepackage{geometry}
\usepackage{lipsum}
\usepackage{multicol}
\usepackage{enumitem}
\usepackage{fancyhdr}
\usepackage[english]{babel}
\usepackage[sc]{mathpazo}                   
\usepackage{graphics}

\usepackage[%  
    colorlinks=true,
    pdfborder={0 0 0},
    linkcolor=red
]{hyperref}

\setlist[itemize]{noitemsep, topsep=0pt}
\setlist[enumerate]{noitemsep, topsep=0pt}

\pagestyle{fancy}
\renewcommand{\headrulewidth}{0pt}
\newcommand{\bb}[1]{\textbf{#1}}


\graphicspath{ {./imgs/} }

\fancyhead[L]{CSCI 545: Introduction to Robotics}
\fancyhead[R]{Fall 2019}

\title{Lecture 17: \it{Motion Planning I}}
\author{Scribe:  \it{Duong Le, Keyu Han, Baiyu Huang, Seunghee Yoon}}
\date{}


\begin{document}
\maketitle
\thispagestyle{fancy}
\section{Complete Motion Planning}
\subsection{Cell Decompose}
\subsection{Visibility Graph}
\section{Sampling-based Motion Planning}
Sampling-based motion planning is not complete: not guarantee to find a solution when one exists. However, as the number of samples goes to infinity, there is a strong guarantee that it can find a solution.\\
There are 2 types of sample-based planning algorithms:\\
\begin{multicols*}{2}
\textbf{\underline{Multiple Query Algorithm}}
\begin{itemize}
\item Roadmap is built beforehand. Then it can be used multiple times for different queries.
\item Query is fast since roadmap is already generated.
\item Keeping roadmap all the time is expensive.
\item Can only deal with static environment - has to rebuild the roadmap when environment changed.
\end{itemize}
\vfill\null
\columnbreak
\textbf{\underline{Single Query Algorithm}}
\begin{itemize}
\item No roadmap is built beforehand. Find a path from start to goal then finish
\item Query is slower.\\
\item No saved roadmap - better memory \\
\item Can deal with dynamic environment 
\end{itemize}
\end{multicols*}

\subsection{Probabilistic Road Map (PRM)}
Probabilistic Roadmap (PRM) is a multi-query algorithm.\\
There are 2 steps:
\begin{enumerate}
\item Preprocessing state: build roadmap
\item Query state: search roadmap given start and goal
\end{enumerate}

\subsubsection{Roadmap Construction}
\begin{figure}[h]
\includegraphics{PRM}
\centering
\caption{Roadmap Construction Algorithm}
\label{fig:roadmap_alg}
\end{figure}
Let $\Delta(q, q')$ be a function that returns either a collision-free path from q to q' or NIL if it cannot find such a path. There are many algorithms can be used for $\Delta$. For example, if we want to check whether there is a straight-line path from q to q, we can use binary search to check for collision.\\
Figure \ref{fig:roadmap_alg} shows an algorithm to construct a roadmap:
\begin{itemize}
\item Line (1) and (2): Initialize graph $G=(V,E)$ to be empty.
\item Line (3) to (8): describes sampling process: keep sampling to get $n$ numbers of collision-free samples.
\item Line (9) to end: describes process to build roadmap from $n$ samples: for every sample node $q\in V$, create a set $N_q$ of $k$ closest neighbors. Then from every $q' \in N_q$, whenever function $\Delta(q, q')$ succeeds to compute a collision-free path from $q$ to $q'$, the edge $(q,q')$ is added to $E$. Figure \ref{fig:roadmap_example} shows an example of roadmap using $\Delta$ as straight-line planner
\end{itemize}

\begin{figure}[h]
\includegraphics{roadmap}
\centering
\caption{An example of a roadmap }
\label{fig:roadmap_example}
\end{figure}

\subsubsection{Query}
When you have a start point and end point, you connect them to the graph and then using graph searching algorithm to find path from start to end.

\subsubsection{Failure}
Some reasons to failure:
\begin{itemize}
\item Connectivity: disconnected graph
\item Obstacles are very closed to the others. 
\end{itemize}
Solution: generate more samples.\\

\begin{figure}
\includegraphics{closed_obs}
\centering
\caption{An example when obstacles are very closed to each other}
\label{fig:closed_obs}
\end{figure}

But in the case when obstacles are closed to the others, the probability to sample between the obstacles is very unlikely. For example, in figure \ref{fig:closed_obs}, the chance to sample points in the narrow road between 2 obstacles is very low. If you keep sampling more points, the graph will become so dense and will take a lot of time to find a path between start and end point.\\
Solution: sample more points toward edges of the obstacles. We can use binary search to find the closest point to the obstacle that's collision-free, and then sample more points near that point. The process of sampling becomes more expensive, but result graph is less dense and faster to compute path.

\subsection{Visibility PRM}
The roadmap is constructed incrementally by randomly sampling the configuration space and attempting to connect some pairs of collition-free samples by the local method.
The visibility roadmaps are build without any explicit computation of the visibility domains.
\begin{figure}[h]
\includegraphics{visibility_psm_visual}
\centering
\caption{node added as a guard; node rejected; connection node merging two connected components}
\label{fig:roadmap_alg}
\end{figure}

\subsection{Principle}
The algorithm that we propose below is general. It allows us to build visibility
roadmaps without requiring any explicit computation of the visibility domains.
The roadmap is constructed incrementally by randomly sampling the conguration
space and attempting to connect some pairs of collision-free samples by the
local method. Figure 2 illustrates the principle of the sampling strategy used at
each iteration of the algorithm. Randomly chosen congurations are checked for
collision to generate samples in $CS_free$; when a free sample is found, it is added to
the roadmap either if it does not ‘see’ any another node of the current roadmap (i.e.
it is a new guard) or if it is seen by at least two nodes belonging to two distinct
connected components of the roadmap (i.e. it is a connection node). The end of
the roadmap’s construction is controlled by a termination condition related to the
volume of free space currently covered by the roadmap.

\subsubsection{Guard and connection node}
When a free sample is found, it is added to the roadmap in two cases:
\begin{itemize}
\item If it does not “see” another node already in R . This will be a new guard.
\item If it is seen at leas by two nodes belonging to two distint connected components of will be a connection node.
\end{itemize}
\subsection{Algorithm}
The algorithm, called Visib-PRM, iteratively processes two sets of nodes: Guard
and Connection. The nodes of Guard belonging to a same connected component
(i.e. connected by nodes of Connection) are gathered in subsets $G_i$.

\begin{figure}[h]
\includegraphics{visibility_psm_code}
\centering
\caption{Visibility PSM Algorithm}
\label{fig:roadmap_alg}
\end{figure}
At each elementary iteration, the algorithm randomly selects a collision-free
configuration q. The main loop processes all the current components $G_i$ of Guard. The algorithm loops over the nodes g in $G_i$ , until it finds a node visible from q. The
first time the algorithm succeeds in finding such a visible node g, it memorizes both
g and its component $G_i$, and switches to the next component $G_i+1$. When q ‘sees’
another guard g
0 in another component $G_j$ , the algorithm adds q to the Connection
set and the component $G_j$ is merged with the memorized $G_i$. If q is not visible from
any component, it is added to the Guard set. The main loop fails to create a new
node when q is visible from only one component; in that case q is rejected.
Parameter ntry is the number of failures before the insertion of a new guard node.
1/ntry gives an estimation of the volume not yet covered by visibility domains.
It estimates the fraction between the non-covered volume and the total volume of
$CS_free$. This is a critical parameter which controls the end of the algorithm. Hence,
the algorithm stops when ntry becomes greater than a user set value M, which
means that the volume of the free space covered by visibility domains becomes
probably greater than (1-1/M).
\subsection{Failure}
\begin{figure}[h]
\includegraphics{545Scribing/visibility_psm_failure}
\centering
\caption{Visibility PSM Failure}
\label{fig:roadmap_alg}
\end{figure}
The random generation of the guards may produce in some cases guards that will
be difficult to connect. This effect is illustrated in Fig. 6 where two guards have
been generated near the boundary of the black triangular obstacle. They fully cover
$CS_free$, however the intersection of both visibility domains is ‘unfortunately’ small.
The only way to complete the roadmap is to pick a connection node in the small
triangle. Then the algorithm will fail if the parameter M is not sufficiently high.Nevertheless this case is only a side-effect of the algorithm. Indeed, in this example,
the probability to select the first two guards with a small intersection domain is very
low. Moreover, this undesirable effect was never observed in practice in all the
examples we experimented on with the algorithm.



\end{document}

